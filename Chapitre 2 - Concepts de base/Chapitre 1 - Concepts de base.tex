\documentclass[10pt]{beamer}
\usepackage[orientation=landscape,size=custom,width=16,height=9,scale=0.5,debug]{beamerposter}
\usepackage[french]{babel}
\usepackage[OT1]{fontenc}
\usepackage{textcomp}
\usepackage{csquotes}
\usepackage{dirtytalk}
\usepackage{float}
\usepackage{framed}
\usepackage{listings}
\usepackage{adjustbox}
\usepackage{array}
\usepackage{epstopdf}
\usetheme[progressbar=frametitle]{metropolis}
\usepackage{appendixnumberbeamer}
\newcommand{\ousommesnous}{Où sommes-nous ?}
\usepackage{hyperref}
\hypersetup{
    colorlinks = true,
    anchorcolor = .,
    linkcolor = .,
    urlcolor = blue,
}
\usepackage{booktabs}
\usepackage[scale=2]{ccicons}
\usepackage{pgfplots}
\usepgfplotslibrary{dateplot}

\usepackage{xspace}
\newcommand{\themename}{\textbf{\textsc{metropolis}}\xspace}

\usepackage{color}
\lstloadlanguages{C,C++,csh,Java}

\definecolor{red}{rgb}{0.6,0,0} 
\definecolor{blue}{rgb}{0,0,0.6}
\definecolor{green}{rgb}{0,0.8,0}
\definecolor{cyan}{rgb}{0.0,0.6,0.6}
\lstset{
language=csh,
basicstyle=\tiny\ttfamily,
numbers=left,
numberstyle=\tiny,
numbersep=5pt,
tabsize=2,
extendedchars=true,
breaklines=true,
frame=b,
stringstyle=\color{blue}\ttfamily,
showspaces=false,
showtabs=false,
xleftmargin=17pt,
framexleftmargin=17pt,
framexrightmargin=5pt,
framexbottommargin=4pt,
commentstyle=\color{green},
morecomment=[l]{//}, %use comment-line-style!
morecomment=[s]{/*}{*/}, %for multiline comments
showstringspaces=false,
morekeywords={ abstract, event, new, struct,
as, explicit, null, switch,
base, extern, object, this,
bool, false, operator, throw,
break, finally, out, true,
byte, fixed, override, try,
case, float, params, typeof,
catch, for, private, uint,
char, foreach, protected, ulong,
checked, goto, public, unchecked,
class, if, readonly, unsafe,
const, implicit, ref, ushort,
continue, in, return, using,
decimal, int, sbyte, virtual,
default, interface, sealed, volatile,
delegate, internal, short, void,
do, is, sizeof, while,
double, lock, stackalloc,
else, long, static,
enum, namespace, string},
keywordstyle=\color{cyan},
identifierstyle=\color{red}
}

\definecolor{main}{RGB}{35,54,59}
\usepackage{caption}
\DeclareCaptionFont{white}{\color{white}}
\DeclareCaptionFormat{listing}{\colorbox{main}{\parbox{\textwidth}{\hspace{15pt}#1#2#3}}}
\captionsetup[lstlisting]{format=listing,labelfont=white,textfont=white, singlelinecheck=false, margin=0pt, font={bf,footnotesize}}


\title{Développement d'applications mobiles - TP}
\subtitle{3° Bachelier en techniques graphiques}
\date{\today}
\author{Daniel Schreurs}
\institute{Haute École de Province de Liège}
%\titlegraphic{\hfill\includegraphics[height=1.5cm]{logo.eps}}

\begin{document}

\maketitle

\setbeamerfont{subsection in toc}{size=\small}
\begin{frame}[allowframebreaks]{Table des matières}
    \setbeamertemplate{section in toc}[sections numbered]
    \tableofcontents
\end{frame}


% \section{Dart}
% \subsection{Présentation}
% \begin{frame}[fragile,t]{\secname : \subsecname}
%     \begin{itemize}
%         \item Publié en 2011 par Google;
%         \item Objectif : remplacer JavaScript;
%         \item Dart peut être compilé en JavaScript;
%         \item Grâce à Flutter, on s'attend à ce qu'il reçoive beaucoup d'attention et de contributions.
%     \end{itemize}
% \end{frame}

% \subsection{Syntaxe et structure}
% \begin{frame}[fragile,t]{\secname : \subsecname}
%     \begin{itemize}
%         \item Orienté objet
%         \item Se situe entre Java et JavaScript.
%         \item Les éléments peuvent être globaux ("top-level") ou attachés à une classe.
%     \end{itemize}
%     \begin{lstlisting}[caption={Class vs main},language=C]
%         void main () {
%             MyNumber number = new MyNumber(42); print(number.value);
%         }
%         class MyNumber {
%             int value; MyNumber(int number) {
%             this.value = number; }
%         }
%     \end{lstlisting}
% \end{frame}

% \begin{frame}[fragile,t]{\secname : \subsecname}
\begin{itemize}
    \item Tout est un objet
    \item $null$ par défaut.
    \item $null$ est une instance de la classe Null.
\end{itemize}
\begin{lstlisting}[caption={Visibilité},language=C]
    void main () {
    MyNumber number = MyNumber(null, (int number) =>
    print(’the number is: ’+ (number?.toString() ?? ’unset’)));

    number.print(); 
    }

    class MyNumber {
        int value;
        Function printer; 
        MyNumber(this.value ,
        void print() => printer(value);
    }
    \end{lstlisting}
%\end{frame}

\end{document}